\documentclass{lab}

\begin{document}

\begin{titlepage}

% \center		% Center everything on the page

\pagestyle{empty}	% Нумерация выкл.

\begin{center}
	\textsc{\LARGE Московский Физико-Технический Институт}\\[1,5cm]
	\textsc{\Large Кафедра общей физики}\\[0,5cm]
	\textsc{\large Лабораторная работа \textnumero 3.2.5}\\[2.5cm]

	\noindent\rule{\textwidth}{1pt}
	\\[0.5cm]
	{ \huge \bfseries Вынужденные колебания в электрическом контуре}
	\\[0.1cm]
	\noindent\rule{\textwidth}{1pt}
\end{center}

\vfill

\begin{minipage}[b]{0.3\textwidth}
	Маршрут \RomanNumeralCaps{3}\\\\
	6 октября 2018 г.\\
	13 октября 2018 г.
\end{minipage}
\hfill
\begin{minipage}[b]{0.33\textwidth}
	\textit{Работу выполнил}\\
	Валиев Ринат, 711 гр.\\\\
	\textit{Под руководством}\\
	Г.И. Лапушкина, к.ф.-м.н.
\end{minipage}

% \vfill % Fill the rest of the page with whitespace

\end{titlepage}

% \newpage
\pagestyle{plain}		% Нумерация вкл.
\setcounter{page}{2}	% Начать нумерацию с номера 1

\section*{Подготовка}

\subsection*{Цель работы}

Исследование вынужденных колебаний и процессов их установления.
В работе используются: генератор звуковой частоты, осциллограф, вольтметр, частотометр,
ёмкость, индуктивность, магазин сопротивлений, универсальный мост.

\subsection*{Теоретическая часть}

В данной работе будем рассматривать колебания в электрическом колебательном контуре под
воздействием внешней ЭДС, гармонически изменяющейся во времени. 
Получаем, что при подключении внешнего источника возникнут колебания, которые будем
рассматривать как решение дифференциального уравнения:
\begin{equation}
L\ddot{I} + R \dot{I} + \dfrac{I}{C} = - \mathscr{E} \Omega \sin \Omega t
\end{equation}
в качестве суперпозиции двух синусоид:
\begin{equation}
I = Be^{-\gamma t} \sin (wt-\theta) + \dfrac{\mathscr{E} \Omega}{L \rho_0} \sin (\Omega t - \psi)
\end{equation}
одна из которых с частотой собственных колебаний контура $\omega$ и амплитудой,
экспоненциально убывающей со временем; вторая - с частотой внешнего источника и постоянной
амплитудой. Однако со временем собственные колебания затухают, и в контуре устанавливаются
вынужденные колебания. А их амплитуда максимальна, когда знаменатель второй синусоиды
$\rho_0 = \sqrt{(\omega_0^2 - \Omega^2_0)^2 + (2\gamma \Omega)^2}$ минимален, то есть
$\omega_0 = \Omega$ (частота внешнего сигнала совпадает с собственной частотой контура).
Это явление и называется \textit{резонансом}. Зависимость амплитуды колебаний от частоты
внешнего напряжения называется \textit{резонансной кривой}.

\subsection*{Резонансная кривая колебательного контура}

\begin{wrapfigure}[10]{l}{6cm}
	\includegraphics[width=5cm]{Scheme2}
	\caption{\footnotesize Схема установки}
	\label{scheme1}
\end{wrapfigure}

Мы можем снять зависимость амплитуды напряжения на резисторе $R$ от частоты на генераторе
(при постоянной амплитуде выходного напряжения), однако для этого выходное сопротивление
генератора должно быть много меньше импеданса контура. Для этого в цепи используется
конденсатор $C_1$. И в таком случае импеданс внешней по отношению к контуру цепи был
гораздо больше импеданса самого контура вблизи резонанса:

$$\dfrac{1}{\omega C_1} \gg |Z_\text{рез}| = \dfrac{L}{RC} $$

\newpage

\subsection*{Процессы установления и затухания колебаний}

\begin{wrapfigure}[12]{l}{7cm}
	\includegraphics[width=7cm]{Scheme3}
	\caption{\footnotesize Нарастание и затухание вынужденных колебаний}
\end{wrapfigure} 

Добротность контура можно определить и другими способами, например, по скорости затухания
свободных колебаний. Подавая на контур цуги синусоид конечной длины, можно наблюдать
процессы установления и затухания колебаний в контуре. И те, и другие могут быть
использованы для определения добротности контура по скорости нарастания/затухания
напряжения:\\

$$\Theta = \dfrac{1}{n} \ln \dfrac{U_0 - U_k}{U_0-U_{k+n}} $$

Измеряя амплитуды напряжения в какой-нибудь момент времени и через n периодов, можем
посчитать добротность по формуле:
$$Q = \dfrac{\pi}{\gamma T} = \dfrac{\pi}{\Theta}$$

\section*{Установка и измерения}

\begin{figure}[!h]
	\includegraphics[width = 0.9 \textwidth]{Scheme}
	\caption{\footnotesize Схема экспериментальной установки для исследования вынужденных
		колебаний}
\end{figure}

Идеальная схема, изображённая на рисунке \ref{scheme1}, не соответствует действительности.
Элементы цепи не идеальны и имеют паразитные сопротивления. Измерим все величины для разных
частот с помощью RLC – моста:

$$R_L = 22.2~Ом, \; L = 99.97~мГн, \; C = 103.33~нФ, \; R = 113.7~Ом$$

\subsection*{Метод резонансных кривых}

Снимем зависимость напряжения на конденсаторе от входной частоты, и получим таким образом
резонансную кривую. Также в таблицу внесем погрешности измерений:

\begin{table}[H]
\centering
\renewcommand{\arraystretch}{1.3}
\resizebox{\textwidth}{!}{
	\begin{tabular}{|c|ccccccccccccccc|}
		\hline
		$ U $, мВ	& 216	& 332	& 475	& 617	& 775	& 949	& 1139	& 1329	& 1139	& 949	& 760	& 617	& 443	& 316	& 240	\\ \hline
		$ f $, кГц	& 1,442	& 1,478	& 1,504	& 1,518	& 1,528	& 1,539	& 1,548	& 1,561	& 1,574	& 1,582	& 1,594	& 1,608	& 1,630	& 1,661	& 1,700	\\ \hline
		$ U/U_0 $	& 0,163	& 0,250	& 0,357	& 0,464	& 0,583	& 0,714	& 0,857	& 1,000	& 0,857	& 0,714	& 0,572	& 0,464	& 0,333	& 0,238	& 0,181	\\ \hline
		$ f/f_0 $	& 0,921	& 0,944	& 0,960	& 0,969	& 0,976	& 0,983	& 0,989	& 0,997	& 1,005	& 1,010	& 1,018	& 1,027	& 1,041	& 1,061	& 1,086	\\ \hline
		$ \Delta~U/U_0 $	& 0,002	& 0,002	& 0,002	& 0,002	& 0,002	& 0,003	& 0,003	& 0,003	& 0,003	& 0,003	& 0,002	& 0,002	& 0,002	& 0,002	& 0,002	\\ \hline
		$ \Delta~f/f_0 $	& 0,002	& 0,002	& 0,003	& 0,003	& 0,003	& 0,003	& 0,003	& 0,003	& 0,003	& 0,003	& 0,003	& 0,003	& 0,003	& 0,003	& 0,003	\\ \hline
	\end{tabular}}
\renewcommand{\arraystretch}{1}
\caption{\footnotesize Полученные значения зависимости амплитудного напряжения от частоты
	для дальнейшего построения АЧХ при R = 0 Ом}
\label{tab1}
\end{table}

\begin{table}[H]
\centering
\renewcommand{\arraystretch}{1.3}
\resizebox{\textwidth}{!}{
\begin{tabular}{|c|cccccccccccccc|}
		\hline
		$ U $, мВ	& 78	& 94	& 108	& 138	& 182	& 246	& 282	& 264	& 228	& 192	& 153	& 126	& 96	& 82	\\ \hline
		$ f $, кГц	& 1,276	& 1,316	& 1,360	& 1,410	& 1,460	& 1,511	& 1,564	& 1,608	& 1,652	& 1,694	& 1,762	& 1,831	& 1,955	& 2,172	\\ \hline
		$ U/U_0 $	& 0,277	& 0,333	& 0,383	& 0,489	& 0,645	& 0,872	& 1,000	& 0,936	& 0,809	& 0,681	& 0,543	& 0,447	& 0,340	& 0,291	\\ \hline
		$ f/f_0 $	& 0,815	& 0,840	& 0,868	& 0,900	& 0,932	& 0,965	& 0,999	& 1,027	& 1,055	& 1,082	& 1,125	& 1,169	& 1,248	& 1,387	\\ \hline
		$ \Delta~U/U_0 $	& 0,001	& 0,001	& 0,001	& 0,001	& 0,001	& 0,001	& 0,001	& 0,001	& 0,001	& 0,001	& 0,001	& 0,001	& 0,001	& 0,001	\\ \hline
		$ \Delta~f/f_0 $	& 0,002	& 0,002	& 0,002	& 0,002	& 0,002	& 0,003	& 0,003	& 0,003	& 0,003	& 0,003	& 0,003	& 0,003	& 0,003	& 0,003	\\ \hline
\end{tabular}}
\renewcommand{\arraystretch}{1}
\caption{\footnotesize Полученные значения зависимости амплитудного напряжения от частоты
	для дальнейшего построения АЧХ при R = 100 Ом}
\label{tab2}
\end{table}

Используя полученные данные построим резонансные кривые для каждой величины сопротивления:

\begin{figure}[H]
	\centering
	\includegraphics[width = 0.9 \textwidth]{Graph}
	\caption{\footnotesize Резонансные кривые для $R = 0$ Ом, $R = 1000$ Ом в приведенных
		координатах; значения из таблиц \ref{tab1} и \ref{tab2}. Также нанесена прямая для
		определения добротности по ширине кривых.}
\end{figure}

\newpage

\subsection*{Метод цугов}

Добротность можно определить и другим способ. Будем посылать на осциллограф синусоидальный
сигнал порциями. Тогда на экране увидим как сигнал нарастает и затухает. В условиях
резонанса огибающая затухающих колебаний -- это перевернутая огибающая нарастающего участка.
Снимем три пары точек для дальнейших вычислений:

\begin{table}[H]
	\centering
	\renewcommand{\arraystretch}{1.3}
	\begin{tabular}{|c|c|c|c|c|c|c|c|c|c|c|c|c|}
		\hline
						& \multicolumn{6}{c|}{Нарастание}						& \multicolumn{6}{c|}{Затухание}						\\ \hline
		$R$, Ом			& \multicolumn{3}{c|}{0}	& \multicolumn{3}{c|}{100}	& \multicolumn{3}{c|}{0}	& \multicolumn{3}{c|}{100}	\\ \hline
		$U_{0}$, дел	& \multicolumn{3}{c|}{39}	& \multicolumn{3}{c|}{40}	& \multicolumn{6}{c|}{-} 								\\ \hline
		$U_{k}$, дел	& 4 	& 7		& 10	& 10	& 20	& 10	& 37	& 36	& 32	& 40	& 31.5	& 40	\\ \hline
		$U_{k+n}$, дел	& 24.5 	& 30	& 35	& 37.5	& 37.5	& 36	& 16.5	& 10	& 5		& 4		& 4		& 6.5	\\ \hline
		n				& 9		& 13	& 20	& 6		& 5		& 5		& 9		& 13	& 20	& 6		& 5		& 5		\\ \hline
		Q 				& 32.1	& 32.2	& 31.7	& 7.6	& 7.6	& 7.8	& 35.0	& 31.9	& 33.8	& 8.2	& 7.6 	& 8.6	\\ \hline
	\end{tabular}
	\caption{\footnotesize Измерение добротности по нарастанию и затуханию порционных сигналов.\\
	Метод описан выше. Формулы для расчетов также в теоретической части.}
	\label{tab3}
	\renewcommand{\arraystretch}{1}
\end{table}

Используем данные таблицы \ref{tab3} и формулы в начале работы для расчета добротности по
скорости нарастания и затухания колебаний. Результаты внесем в таблицу \ref{tab4}.

\subsection*{Погрешности}

Погрешности измерений и вычислений определим через параметры приборов и по шкале
осциллографа. Приведем лишь некоторые формулы расчета погрешностей:

Для погрешности теоретического вычисления погрешности используем:

\begin{equation}
\begin{aligned}
	&\Delta_Q = \sqrt{\left( \dfrac{\partial Q}{\partial R} \right)^2 \cdot \Delta_R^2 +
	\left( \dfrac{\partial Q}{\partial L} \right)^2 \cdot \Delta_L^2 + \left( \dfrac{\partial Q}{\partial C} \right)^2 \cdot \Delta_C^2} \\
	&\Delta_Q = \sqrt{\dfrac{L}{R^4C} \cdot \Delta_R^2 + \dfrac{1}{4R^2LC} \cdot \Delta_L^2 + \dfrac{L}{4C^3R^2} \cdot \Delta_C^2}
\end{aligned}
\end{equation}

В случае вычисления добротности через затухающую часть графика имеем:

\begin{equation}
\begin{aligned}
	&Q = \dfrac{\pi}{\Theta} , ~~~ где \ \Theta = \frac{1}{n} \ln (\dfrac{U_k}{U_{k+n}})\\
	&\Delta_{\Theta} = \sqrt{\left( \dfrac{\partial {\Theta}}{\partial U_k} \right)^2 \cdot \Delta_{U_k}^2 +
	\left( \dfrac{\partial {\Theta}}{\partial U_{k+n}} \right)^2 \cdot \Delta_{U_{k+n}}^2}\\
	&\Delta_Q = \pi \cdot \frac{\Delta_{\Theta}}{\Theta^2}
\end{aligned}
\end{equation}

\section*{Итоги}

В работе были определены добротности контуров с и без дополнительного сопротивления
$ R_{100} $ разными способами.

\begin{table}[H]
	\centering
	\renewcommand{\arraystretch}{1.3}
	\begin{tabular}{|c|c|c|c|c|}
		\hline
		& Теория & Резонансная кривая & Нарастание & Убывание     \\ \hline
		$Q_{R=0}$	& 36.2	& $34.5 \pm 1.2$	& $32 \pm 4$	& $33.6 \pm 3$ \\ \hline
		$Q_{R=100}$	& 7.0	& $6.2 \pm 0.4$		& $7.6 \pm 1$	& $8.1 \pm 1$  \\ \hline
	\end{tabular}
	\caption{\footnotesize Сравнение экспериментальных значений добротности, полученных разными методами. Каждый метод описан выше.}
	\label{tab4}
	\renewcommand{\arraystretch}{1}
\end{table}

В целом добротности оказались примерно одинаковыми при измерении разными способами. Однако,
результаты немного разнятся. Следует заметить, что магазин сопротивлений мог дать
значительный вклад для сопротивления в контуре с $ R = 0 $ Ом, т.к. резисторы собраны в
виде катушек. Данный факт не был учтен в работе.

\subsubsection*{Замечание}
Рассмотрим процесс установления колебаний в контуре с высокой добротностью вблизи резонанса.
Этот процесс описывается при начальных условиях $ (U = 0, \ddot{U} = 0) $ формулой:
$$ U = U_0[\cos(\Omega t - \psi) - \exp^{\gamma t} \cos (\omega_0 t - \psi)] $$
Напряжение содержит два близких по частоте колебания, между которыми происходят биения.
Появление биений связано с тем, что разность фаз этих колебаний медленно меняется; при
нулевой разности фаз они вычитаются друг из друга, а при расхождении фаз на $ \pi $ --
складываются. Амплитуда колебаний то растет, то падает, испытывая биения. При порционных
сигналах также наблюдаются схожая картина.

\begin{figure}[!h]
	\centering
	\includegraphics[width = 0.5 \textwidth]{cool}
	\caption{\footnotesize Биение колебаний вблизи резонанса}
\end{figure}

\end{document}